%Conditional to render comments conditional on the command \supresscomments being undefined
\ifx\supresscomments\undefined%
  \newcommand{\cole}[1]{%
    \noindent{\color{blue} \textsf{[CC: #1]}}
  }
  \newcommand{\titouan}[1]{%
    \noindent{\color{red} \textsf{[TC: #1]}}
  }
    \newcommand{\robert}[1]{%
    \noindent{\color{purple} \textsf{[RIB: #1]}}
  }
\else%
  \newcommand{\robert}[1]{}
  \newcommand{\cole}[1]{}
  \newcommand{\titouan}[1]{}
\fi


%Prefixes and strings.  Do not use these!
\newcommand{\Aff}{%
  \mathsf{Aff}
}
\newcommand{\Mat}{%
  \mathsf{Mat}
}
\newcommand{\Lin}{%
  \mathsf{Lin}
}
\newcommand{\Lag}{%
  \mathsf{Lag}
}
\newcommand{\Co}{%
  \mathsf{Co}
}
\newcommand{\Isot}{%
  \mathsf{Isot}
}

\newcommand{\zx}{%
  \mathsf{GSA}
}
\newcommand{\gla}{%
  \mathsf{GLA}
}
\newcommand{\gaa}{%
  \mathsf{GAA}
}
\newcommand{\stab}{%
  \mathsf{Stab}
}
\newcommand{\GGA}{%
  \mathsf{GGA}
}
\newcommand{\GQGA}{%
  \GSA[\C]^+
}
\newcommand{\Gauss}{\mathsf{Gauss}}
\newcommand{\QGauss}{\mathsf{QGauss}}
\newcommand{\ExGauss}{\operatorname{ExGauss}}
\newcommand{\GaussRel}{\mathsf{GaussRel}}
\newcommand{\QGaussRel}{\ALR[\C]^+}


%%%%%%%%%%%%%%%%%%%%%%%%%%%%%%%%%%%%%%%%%%%%%%%%%%%%%%%%%%%%%%
%%%%%%%%%%%%%%%%%%%%%%%%%%%%%%%%%%%%%%%%%%%%%%%%%%%%%%%%%%%%%%
%%%%%%%%%%%%%%%%%%%%%%%%%%%%%%%%%%%%%%%%%%%%%%%%%%%%%%%%%%%%%%
%%%%%%%%%%%%%%%%%%%%%%%%%%%%%%%%%%%%%%%%%%%%%%%%%%%%%%%%%%%%%%

%Functors and math operators
\newcommand{\im}{%
  \operatorname{im}
}
\newcommand{\CPM}{%
  \operatorname{CPM}
}
\newcommand{\interp}[1]{%
  \left\llbracket #1 \right\rrbracket
}
\newcommand{\trans}{%
  \mathsf{T}
}
\DeclareRobustCommand{\disc}{%
  {\scalebox{.5}{\tikzfig{../figures/assets/discard_small}}}
}

%Rings
\newcommand{\B}{%
  \mathbb{B}
}
\newcommand{\N}{%
  \mathbb{N}
}
\newcommand{\Z}{%
  \mathbb{Z}
}
\newcommand{\Zp}{%
  {\mathbb{F}_p}
}
\newcommand{\K}{%
  \mathbb{K}
}
\newcommand{\Q}{%
  \mathbb{Q}
}
\newcommand{\R}{%
  \mathbb{R}
}
\newcommand{\C}{%
  \mathbb{C}
}
\renewcommand{\H}{%
  \mathbb{H}
}
\renewcommand{\O}{%
  \mathbb{O}
}

%Categories
\newcommand{\Coherent}{%
  \mathsf{Coherent}
}
\newcommand{\ExCoherent}{%
  \mathsf{Ex}\Coherent
}
\newcommand{\LOv}{%
  \mathsf{LOv}
}
\newcommand{\ECirc}{%
  \mathsf{ECirc}
}

%Categories with default parameters
\newcommand{\FHilb}{\mathsf{FHilb}}
\newcommand{\Hilb}{\mathsf{Hilb}}

\NewDocumentCommand{\Rel}{O{X}}{%
  \mathsf{Rel}_{#1}
}
\NewDocumentCommand{\Symp}{O{\K}}{%
  \mathsf{Symp}_{#1}
}
\NewDocumentCommand{\ASymp}{O{\K}}{%
  {\Aff}\Symp[#1]
}
\NewDocumentCommand{\AR}{O{\K}}{%
  {\Aff}\Rel[#1]
}
\NewDocumentCommand{\lR}{O{\K}}{%
  {\Lin}\Rel[#1]
}
\NewDocumentCommand{\LR}{O{\K}}{%
  {\Lag}\Rel[#1]
}
\NewDocumentCommand{\IR}{O{\K}}{%
  {\Isot}\Rel[#1]
}
\NewDocumentCommand{\CR}{O{\K}}{%
  {\Co}\IR[#1]
}
\NewDocumentCommand{\ALR}{O{\K}}{%
  {\Aff}\LR[#1]
}
\NewDocumentCommand{\AIR}{O{\K}}{%
  {\Aff}\IR[#1]
}
\NewDocumentCommand{\ACR}{O{\K}}{%
  {\Aff}\CR[#1]
}
\NewDocumentCommand{\ZX}{O{\K}}{%
  \zx_{#1}
}
\NewDocumentCommand{\GSA}{O{\K}}{%
  \zx_{#1}
}
\NewDocumentCommand{\ZXdisc}{O{\K}}{%
  \ZX[#1]^\disc
}
\NewDocumentCommand{\GLA}{O{\K}}{%
  \gla_{#1}  
}
\NewDocumentCommand{\GAA}{O{\K}}{%
  \gaa_{#1}
}
\NewDocumentCommand{\Stab}{O{p}}{%
  \stab_{#1}
}


%Categories with fixed parameters
\newcommand{\ZXF}{%
  \ZX[\Zp]
}
\newcommand{\RX}{%
  \Rel[X]
}
\newcommand{\RK}{%
  \Rel[\K]
}

%abbreviations, to be used in interpretations
\newcommand{\abbrlinr}{%
  \mathsf{LR}
}
\newcommand{\abbrar}{%
  \mathsf{AR}
}
\newcommand{\abbrlagr}{%
  \mathsf{LR}
}
\newcommand{\abbrisotr}{%
  \mathsf{IR}
}
\newcommand{\abbrcoisotr}{%
  \mathsf{CIR}
}
\newcommand{\abbralagr}{%
  \mathsf{ALR}
}
\newcommand{\abbraffisotr}{%
  \mathsf{AIR}
}
\newcommand{\abbraffcoisotr}{%
  \mathsf{ACIR}
}

% Grassmanians and homs
%matrix hom: usage   \Matrices[default=m,default=n,\default=\K] produces M_{m,n}(K) 
\NewDocumentCommand{\Matrices}{ O{m} O{n} O{\K} }{%
%  \operatorname{\Mat}_{#3}(#2,#1)
  \operatorname{M}_{#1,#2}(#3)
}
%Symmetric matrices:  usage   \Matrices[default=n,\default=\K] produces Sym_{n}(K) 
\NewDocumentCommand{\Sym}{O{n} O{\K} }{%
  \operatorname{Sym}_{#1}(#2)
}

%Unitary matrices:  usage   \Matrices[default=n,\default=\C] produces U_{n}(K) 
\NewDocumentCommand{\Unit}{O{n} O{\C} }{%
  \operatorname{U}_{#1}(#2)
}

%Orthogonal matrices:  usage   \Matrices[default=n,\default=\R] produces O_{n}(K) 
\NewDocumentCommand{\Orth}{O{n} O{\R} }{%
  \operatorname{O}_{#1}(#2)
}

%Special Orthogonal matrices:  usage   \Matrices[default=n,\default=\R] produces O_{n}(K) 
\NewDocumentCommand{\SpOrth}{O{n} O{\R} }{%
  \operatorname{SO}_{#1}(#2)
}

%Symplectic matrices:  usage   \Matrices[default=n,\default=\R] produces Sp_{n}(K) 
\NewDocumentCommand{\Sp}{O{n} O{\R} }{%
  \operatorname{Sp}_{#1}(#2)
}

\NewDocumentCommand{\ASp}{O{n} O{\R} }{%
  \operatorname{AffSp}_{#1}(#2)
}


%Linear grassmanian:  usage   \Matrices[default=n,\default=\R] produces Sp_{n}(K) 
\NewDocumentCommand{\Gl}{O{n} O{\R} }{%
  \operatorname{Gl}_{#1}(#2)
}

% Diagram symbols
\newcommand{\bvdots}{%
  \tikz[baseline, every node/.style={inner sep=0}]{ \node at (0,0){.}; \node at (0,-6pt){.}; \node at (0,6pt){.}; }
}

%\cole{...}
\newcommand{\diagram}{%
  \(\ZX\)-diagram\xspace
}

%macros
\newlength\oversetwidth
\newlength\underwidth
\newcommand\alignedoverset[2]{
  % #1 = over
  % #2 = under
  \settowidth\oversetwidth{$\overset{#1}{#2}$}
  \settowidth\underwidth{$#2$}
  \setlength\oversetwidth{\oversetwidth-\underwidth}
  \hspace{.5\oversetwidth}
  &
  \settowidth\oversetwidth{$\overset{#1}{#2}$}
  \settowidth\underwidth{$#2$}
  \setlength\oversetwidth{\oversetwidth-\underwidth}
  \hspace{-.5\oversetwidth}
  \overset{#1}{#2}
}

\newcommand{\stackedrefs}[1]{%
  \begingroup\renewcommand*{\arraystretch}{.5}\begin{matrix}#1\end{matrix}\endgroup
}

\newcommand{\stackeqmid}[1]{%
  \stackrel{\stackedrefs{#1}}{=}
%  \stackeq{#1}
}
\newcommand{\stackeq}[1]{%
%  \stackrel{\mathllap{\stackedrefs{#1}}}{=}
  \alignedoverset{\stackedrefs{#1}}{=}
}

\let\oldoverline\overline
\renewcommand{\overline}[1]{\mkern 1.5mu\oldoverline{\mkern-1.5mu#1\mkern-1.5mu}\mkern 1.5mu}


%redefine ugly epsilon and phi
\renewcommand{\phi}{%
  \varphi
}
\renewcommand{\epsilon}{%
  \varepsilon
}


\DeclareMathOperator{\diag}{diag}
\DeclareMathOperator{\rk}{rank}

%\renewcommand{\Re}{%
%  \mathfrak{Re}
%}
%
%\renewcommand{\Im}{%
%  \mathfrak{Im}represent
%}

%for standardization
\renewcommand{\leq}{%
  \leqslant
}
\renewcommand{\geq}{%
  \geqslant
}

%
\renewcommand{\overrightarrow}{%
  \vec
}


\newcommand{\vdotss}{ 	\tikz[baseline, every node/.style={inner sep=0}]{ 	\node at (0,0){.}; 	\node at (0,4pt){.}; 	\node at (0,8pt){.}; 	} 	}
\newcommand{\vdotsn}{{\ \tikz[baseline, every node/.style={inner sep=0}]{\node at (0,.23){$\vdots$}; \node at (.2,.1){$\scriptstyle n$}}\!}}
\newcommand{\vdotsm}{{\ \tikz[baseline, every node/.style={inner sep=0}]{\node at (0,.23){$\vdots$}; \node at (.2,.1){$\scriptstyle m$}}\!}}